\documentclass[a4paper,11pt]{article}
\usepackage[utf8]{inputenc}
\usepackage[T1]{fontenc}
\usepackage[english]{babel}
\usepackage{geometry}
\geometry{margin=2.5cm}
\usepackage{setspace}
\usepackage{titlesec}
\usepackage{hyperref}
\usepackage{xcolor}
\usepackage{graphicx}

\titleformat{\section}{\Large\bfseries\color{cyan}}{}{0pt}{}
\titleformat{\subsection}{\bfseries\color{teal}}{}{0pt}{}

\title{\textbf{Oral Presentation – Olafur Eliasson: \textit{Ice Watch}}}
\author{Binôme Presentation – 5 minutes}
\date{}

\begin{document}

\maketitle
\onehalfspacing

\section*{Introduction (Slide 1)}
\textbf{Person 1:} \\
Today we’re going to talk about an artist who uses art to denounce the ecological crisis: \textbf{Olafur Eliasson}.  
His work \textit{Ice Watch} shows how art can make people see and feel the effects of climate change.  
Instead of using words or data, he uses real ice to create emotion.

\section*{The Artist (Slide 2)}
Olafur Eliasson is a Danish–Icelandic contemporary artist.  
He often works with light, water, and natural elements.  
His main goal is to make people aware of how nature and humans are connected.  
He mixes art, science, and activism to inspire reflection and change.  
For him, art is not decoration — it’s a message.

\section*{The Artwork (Slide 3)}
In 2014, Eliasson created \textit{Ice Watch}, a huge environmental installation.  
He brought twelve massive blocks of ice from a fjord in Greenland and placed them in a circle, like a clock, in the center of the city.  
The public could touch the ice, feel the cold, and watch it melt day after day.  
The experience was powerful because people could literally see climate change happening in front of them.  
It was an art piece you could feel with your senses.

\section*{Where and When (Slide 4)}
The installation was first shown in \textbf{Copenhagen (2014)}, then in \textbf{Paris (2015)} during the \textbf{COP21}, and finally in \textbf{London (2018)}.  
Each place was symbolic — big cities where decisions about the environment are discussed.  
By showing his work outside, Eliasson reached thousands of people who weren’t expecting to see art in the street.  
It turned the city itself into a museum about the planet.

\bigskip
\textit{\textbf{→ End of Part 1 (Person 1)}}

\newpage

\section*{The Message (Slide 5) – Person 2}
The message of \textit{Ice Watch} is simple but strong: the ice is melting, just like our planet.  
When people see the ice disappear, they realize that climate change isn’t a distant problem — it’s happening now.  
Eliasson wanted people to understand that we are all part of the environment, not separated from it.  
It’s not just about science — it’s about emotion and empathy.

\section*{Type of Art (Slide 6)}
This kind of art is called \textbf{environmental installation art}.  
It’s temporary, public, and interactive — the audience is part of the work.  
What makes \textit{Ice Watch} unique is that the art actually melts and disappears.  
That disappearance represents the loss of our natural world if we do nothing.

\section*{Impact and Meaning (Slide 7)}
The installation had a big impact.  
It was covered by media all over the world and made people talk about global warming in a new way.  
Scientists supported the project, saying it helped the public understand the reality behind their data.  
Eliasson showed that art and science can work together.  
His project transformed something invisible — climate change — into a visible and emotional experience.

\section*{Conclusion (Slide 8)}
In conclusion, \textit{Ice Watch} is more than just an artwork — it’s a warning and a call to action.  
Olafur Eliasson makes us reflect on our role in protecting the planet.  
He proves that art can change how people see the world.  
By using beauty and simplicity, he creates awareness and emotion — two things we need to fight climate change.

\bigskip
\textit{\textbf{→ End of Part 2 (Person 2)}}

\end{document}
